\renewcommand{\konferenztag}{\dienstag}
\newtimeslot{09:00}
\abstractGruen{Pirmin Kalberer}%
{GeoPackage, das Shapefile der Zukunft}%
{}%
{Im Februar 2014 hat das Open Geospatial Consortium den GeoPackage Encoding Standard offiziell
freigegeben. Dieses noch junge Format hat sich bereits gut etabliert und wird in zahlreichen GIS-Produkten unterstützt.
In Geo\-Package-Dateien können sowohl Vektor- als auch Rasterdaten mit zugehörigen Metainformation gespei\-chert
werden. Damit können Geodaten einfach ausgetauscht und auch auf mobilen Geräten effizient genutzt werden.}

\abstractGiStudio{Christopher Lorenz}%
{osm\_address\_db -- Adressdaten in der OSM-Datenbank}%
{}%
{Das Projekt osm\_address\_db ermöglicht die Aufbereitung der in OSM vorhandenen Adressdaten. Dazu wurden Shell-
und SQL-Skripte entwickelt, die eine Tabellenstruktur zur Auswertung und Analyse aufbauen.
Im Vortrag wird die Datenstruktur, die benutzten Tools und Technologien als auch die während der
Entwicklung aufgetretenen Probleme eingegangen. Die OSM-Daten werden nach dem Karlsruher Schema analysiert,
dabei finden auch associated\-Street-Relationen Berücksichtigung.
Derzeit befindet sich eine Webanwendung zur Auswertung in Entwicklung.}

\newtimeslot{09:30}
\abstractGruen{Michael Paulmann}%
{Umstellung von Übungen auf QGIS}%
{}%
{Im Modul Grundlagen Geographischer Informationssysteme an der Hochschule für Technik in Stuttgart
sollen die Übungen und die Vorlesung nicht mehr in einem proprietären Geoinformationssystem
stattfinden, sondern in die Open-Source-Software QGIS. Es gibt einen Einblick in die Probleme
die bisher auftraten und den bisherigen Fortschritt, der Prozess ist noch nicht abgeschlossen.
Bis Oktober 2016 soll alles auf QGIS geändert worden sein.}

\abstractGiStudio{Manuel Roth}%
{Vector Tiles from OpenStreetMap}%
{}%
{Das Projekt OSM2VectorTiles bietet einerseits einen Workflow, um selbst Vektor Tiles aus OSM zu erstellen,
und bietet diese andererseits gratis zum Download an. Dies ermöglicht jedem, eine selbst gestaltete Karte der
gesamten Welt zu erstellen und anzubieten, ohne Kenntnisse von PostGIS und Mapnik zu haben.
Die Zuhörer wissen nach unserer Präsentation, wie sie eine selbst gestaltete Karte erstellen
und veröffentlichen können. Zudem können sie die vom OSM2VectorTiles-Projekt zur Verfügung gestellten
Daten verwenden, um ihre Karte auch offline anbieten zu können.}

\newtimeslot{10:00}

\abstractGruen{Manfred Egger}%
{Automatische Erkennung der Projektion von Geodaten}%
{}%
{Das GIS-Tool Shapefile ProjectionFinder wird als Lösungsvorschlag
in Zusammenhang mit Geodaten und
unbekannter Projektion präsentiert. Zielgruppe dieses Programms sind vor
allem Anwender, die mit geringen
Fachkenntnissen zu Koordinatensystemen gelieferte Daten ohne Aufwand in
GIS-Projekte lagerichtig integrieren möchten.}

\abstractGiStudio{Dietmar Seifert}%
{Stand der Hausnummern in OSM und Hausnummerauswertung auf regio-osm.de}%
{}%
%%%% Text von Michael geschrieben
{Es wird der aktuelle Stand der Hausnummernerfassung in OSM in Deutschland, aber auch in anderen EU-Ländern erläutet.
In einigen Ländern wurden die Hausnummern aus verfügbaren Quellen importiert
oder es gibt landesweite Listen zum Datenabgleich, in Deutschland ist die
Situation je nach Bundesland sehr unterschiedlich und in Bewegung.
Die Auswertungsmöglichkeiten auf regio-osm.de werden vorgestellt.}

\newtimeslot{11:00}

\abstractGruen{Andreas Schmid}%
{Betrieb von QGIS in einer hete\-rogenen Client-Server-Umgebung}%
{}%
{Das Amt für Geoinformation des Kantons Solothurn betreibt für die Benutzer der
Kantonsverwaltung QGIS auf einem Linux-Applikationsserver. Für die Integration
der Linux-Anwendung in den via Citrix bereitgestellten Windows"-Desktop, der den
Standard-Arbeitsplatz in der Kantonsverwaltung darstellt, kommt die Open-Source-Lösung X2Go zum Einsatz.
Der Vortrag präsentiert diese Lösung im Detail und geht auf deren Vor- und
Nachteile ein. Zudem vergleicht er sie mit anderen Lösungen, die früher im Einsatz waren und nun abgelöst wurden.}

\abstractGiStudio{Lars Schimmer}%
{Mapillary-Alltag}%
{}%
{Das Ökosystem Mapillary wurde auf der Fossgis 2014 in einem Lightning
Talk schon vorgestellt.
Ich werde aus Benutzersicht meine Erfahrungen aus 18 Monaten Aktivität mit
Mapillary berichten, dieses umfasst über 160\,000 eingereichte Fotos,
diverse genutzte Mobilgeräte, viele besuchte Orte und mehrere Methoden
der Datenbearbeitung. Natürlich werden auch die Pitfalls und Probleme angesprochen, über die
man zwangsweise stolpert.}

\newtimeslot{11:30}

\abstractGruen{Sara Biesel}%
{QGIS meets MapProxy}%
{}%
{Das Bundesamt für Kartographie und Geodäsie entwickelt ein System, um Kartenausschnitte zu speichern und
die Karten offline im Geoinformationssystem QGIS zu laden. Dabei wird ein
Proxyserver MapProxy verwendet, der bei bestehender Internetverbindung
die Karten als Kacheln lokal ablegt. Das ganze System liegt auf einer
portablen, externen Festplatte.}

\abstractGiStudio{Serhan Şen}%
{Leitstellensimulator goes \mbox{OpenStreetMap}}%
{}%
{Das nichtkommerzielle Browserspiel LstSim, ein Rettungsleitstellensimulator, setzte jahrelang auf die Dienste
der Google Maps API. Leider gab es mit der API über einen längeren Zeitraum anhaltende Schwierigkeiten. Es wurde
deshalb schon früh der Vorschlag gemacht, man solle doch "`einfach auf OpenStreetMap wechseln"'. Doch was bedeutet
das überhaupt und wie sieht so ein Wechsel aus? Dieser Vortrag fasst die technischen und organisatorischen
Herausforderungen -- und Chancen -- seit Beginn der Umstellungen Ende 2015 zusammen.}

\newtimeslot{12:00}
\abstractGruen{Jörg Habenicht}%
{Die Zugriffszeit auf den QGis-Mapserver um Faktor 100 beschleunigen.}%
{}%
{Wir benutzen den QGIS-Mapserver, um den Kunden verschiedene Karten im
Webbrowser anzuzeigen. Mit unseren großen Konfigurationsdateien von bis
zu 19\,MB benötigt der Server beim Erstzugriff zur Berechnung länger als
der Browsertimeout. Mit unserem Tool wird der Timeout verhindert und die
Zugriffszeit 100-fach beschleunigt.}

\abstractGiStudio{Numa Gremling}%
{Turf.js -- Geoverarbeitung im \mbox{Browser}}%
{}%
{Turf.js ist eine Open-Source-JavaScript-Bibliothek, die geographische Analysen und Abfragen ermöglicht.
Im Gegensatz zu Web Processing Services (WPS), die eine komplexe serverseitige Infrastruktur erfordern,
arbeitet Turf.js clientseitig mit Daten im GeoJSON-Format. Turf.js ist somit eine schnell und leicht
umzusetzende Alternative zu komplexen Web-GIS-Lösungen.
Im Vortrag werden klassische Werkzeuge zur Analyse von Geodaten, wie Puffer und Punkt-in-Polygon, aber auch
die Durchführung komplexer Analysen vorgestellt.}

\newtimeslot{13:30}
\abstractGruen{Jonas Eberle}%
{Web-basierte Geoprozessierung \newline mit Python und PyWPS}%
{}%
{Web-basierte Prozessierungsdienste können für eine Vielzahl an Aufgaben verwendet werden und bieten enorme Möglichkeiten,
nutzer- und entwicklerfreundliche Geodatendienste aufzubauen. Dies wird an einigen einfachen und komplexen Beispielen gezeigt,
die mit der Software PyWPS im Rahmen der Programmiersprache Python online veröffentlicht worden sind.
Dieser Vortrag liefert anfangs eine Einführung in den Bereich der „Web Processing Services“ und zeigt
zudem das Potential web-basierter Prozessierung mittels WPS im Geodatenbereich auf. }

\abstractGiStudio{Frederik Ramm}%
{OpenStreet mal ohne Map}%
{}%
{Dieser Vortrag erzählt von verschiedenen -- realistischen wie auch unrealistischen -- Anforderungen,
die potentielle Nutzer abseits von der Kartenerstellung an OpenStreetMap stellen, und skizziert Lösungen und Probleme.}

\newtimeslot{14:00}
\abstractGruen{Arne Schubert}%
{Hybride mobile App-Entwicklung mit Angular}%
{}%
{\dots\ Der Vortrag gibt zunächst eine kurze Einführung in die Grundbegriffe
der hybriden App-Entwicklung und Angular. Es wird ein Grundsystem für mobile
App-Entwicklung erstellt und um Webmapping-Elemente, durch die Angular Leaflet Directive,
erweitert. Das Ergebnis kann dann in einer Vorführanwendung direkt betrachtet werden.

Da das Thema hybride App-Entwicklung sehr Umfangreich ist, ist der Vortrag als eine
Einführung in die Thematik gedacht, die die Vorzüge von hybriden Webmapping-Apps mit
Angular aufzeigen soll.}

\abstractGiStudio{Joachim Kast}%
{Bezahlte und organisierte Edits -- Vorteile und Gefahren für OSM}%
{}%
{OpenStreetMap wird aufgrund Aktualität, Detailtreue und verfügbaren Rohdaten
zunehmend im kommerziellen Umfeld verwendet. Neben der reinen
Datennutzung verändern diese Anwender nun auch verstärkt den
Datenbestand durch Änderungen und Neueintragungen speziell für Ihre
Geschäftsinteressen. Eigentlich gut für das Projekt, aber leider kommt
es oftmals zu erheblichen Missverständnissen und Konflikten.}

\newtimeslot{14:30}
\abstractGruen{Florian Ledermann}%
{mapmap.js -- eine kartographische API für interaktive thematische Karten}%
{}%
{mapmap.js ist eine an der Technischen Universität Wien entwickelte API und
versucht den kartographischen
Visualisierungsprozess in seiner Gesamtheit in einem High"=Level"=Java\-Script"=API
abzubilden, wobei jeder Teilaspekt zunächst "`einfach funktioniert"', bei
Bedarf jedoch im Detail an die Notwendigkeiten der jeweiligen Anwendung angepasst werden
kann und somit auch die Entwicklung neuer Visualisierungstechniken erlaubt.}

\abstractGiStudio{Sven Geggus}%
{Ansätze zur Lokalisierung einer Openstreetmap basierten \mbox{Weltkarte}}%
{}%
{Der Vortrag stellt eine vom Renderer unabhängige Methode zur Latinisierung
von OSM-basierten Karten vor. Als Datenquelle dient, wenn möglich, OSM
selbst. Alternativ wird Transkription verwendet, die jedoch viele
Schriftsystem"=abhängige Probleme birgt, für die teilweise Lösungen
aufgezeigt werden. Ferner wird auf politische Probleme bei der Lokalisierung
von Karten eingegangen.}

\newtimeslot{15:30}
\abstractGruen{Marc Jansen}%
{OpenLayers 3 -- Stand, Neues und Ausblick}%
{}%
{\dots\ Zunächst wird der Vortrag OpenLayers kurz vorstellen, um dann einen Schwerpunkt auf
die Änderungen seit März 2013 zu legen. Abschließend soll kurz dargestellt werden,
welche zukünftigen Entwicklungen derzeit bearbeitet werden und wie der Stand von
häufig nachgefragten Features ist. Wann ist etwa mit der Unterstützung von
LineStrings und Polygonen im WebGL-Renderer zu rechnen? Wann (falls überhaupt)
wird OpenLayers einen vollfunktionalen SLD-Parser bereitstellen? Und was bleibt
noch an der Bibliothek zu entwickeln, nun da sogar Raster\-daten im Browser zur
Laufzeit reprojiziert werden?}

\abstractGiStudio{}%
{OSM Lightning Talks}%
{}%
{Folgender Lightning Talk ist bereits fest eingeplant:
	\begin{itemize}
		\RaggedRight
		\setlength{\itemsep}{-2pt} % Aufzählungspunktabstand auf 0
		\item \emph{Stefan Keller}: Kort Game Reloaded -- Der Spaß geht weiter!
	\end{itemize}
	\justifying
Die anderen Lightning Talks werden auf der Konferenz bekanntgegeben.
}

\newtimeslot{16:00}
\abstractGruen{Tobias Sauerwein}%
{Faster, smaller, better -- Compiling your application together with OpenLayers 3}%
{}%
{OpenLayers 3 setzt den Closure Compiler ein, um Java\-Script in besseres JavaScript zu kompilieren.
Der von Google entwickelte Closure Compiler macht weit mehr als normale Code-Minifier.
Es werden nicht nur Variablen- und Funk\-tionsnamen gekürzt, durch die statische Analyse
des Codes werden eine Reihe von Optimierungen durchgeführt, wie z.\,B. das Entfernen
von nicht verwendetem Code oder Funktions-Inlining. Besonders interessant ist das
Type-Checking und auch ein Syntax-Check, so dass viele Fehler, die sonst erst während
der Ausführung auffallen würden, schon früh entdeckt werden~\dots}

\abstractGiStudio{Lars Roskoden}%
{Das ist ja wohl die Höhe!}%
{}%
{Darstellung einfacher Möglichkeiten für die Höhenmessung von OSM-Objekten. Vergleich der Vor- und Nachteile.}

\newtimeslot{16:30}
\abstractGruen{Marc Jansen}%
{GeoExt3}%
{}%
{Der Vortrag stellt GeoExt 3 vor und wird auch die Vater-Bibliotheken ExtJS
und OpenLayers erläutern. Die Schwer\-punkte werden hier zunächst allgemeine
Features der Bibliotheken/Frameworks sein, bevor der Fokus auf der Erstellung
von "`universalen"' WebGIS-Applikationen liegt.

Unter universal verstehen wir
hierbei eine GeoExt3-basierte Applikation, die sowohl auf klassischen
Desktop-Browsern aber auch auf mobilen Endgeräten wie Tablets und Smartphones
funktioniert und ein ansprechendes Benutzererlebnis ermöglicht~\dots}

\abstractGiStudio{Peter Karich}%
{Flexible Routenplanung mit GraphHopper}%
{}%
{GraphHopper ist ein schneller und flexibler Open-Source-Routenplaner basierend auf OpenStreetMap-Daten,
der sowohl offline auf dem Gerät als auch auf dem Server läuft und schon bei vielen bekannten Organisationen
und Firmen produktiv eingesetzt wird. GraphHopper ermöglicht nicht nur Routing von A nach B, sondern auch
Reichweitenanalyse, Map-Matching und vieles mehr.}

\newtimeslot{17:00}
\abstractGruen{Astrid Emde}%
{Datenerfassung und Suchen mit Mapbender\,3}%
{}%
{Der Vortrag zu Mapbender\,3 stellt fortgeschrittene Elemente mit erweiterter Konfiguration vor. 
\begin{itemize}
\RaggedRight
\setlength{\itemsep}{-2pt} % Aufzählungspunktabstand auf 0
 \item Möglichkeiten der Datenerfassung mit dem Mapbender\,3 Digitizer
 \item Aufbau von Suchen mit dem SearchRouter (SQL-basierte Suchen)
 \item Aufbau von Suchen mit SimpleSearch (Solr-basierte Suchen)
\end{itemize}
\justifying%
}

\abstractGiStudio{Bernhard Ströbl}%
{XPlanung für einen Flächennutzungsplan mit PostGIS und QGIS}%
{}%
{Der Vortrag zeigt die erfolgte Umsetzung des Standards XPlanung für PostGIS und den Zugriff darauf aus
QGIS heraus am Beispiel eines in der Aufstellung befindlichen Flächennutzungsplans. }

\newtimeslot{17:30}\label{bof-dienstag}
\abstractGruen{Astrid Emde}%
{Mapbender\,3-Anwendertreffen}%
{}%
{Anwender und Entwickler der WebGIS-Client-Suite Mapbender sind zu diesem Treffen
eingeladen. Der aktuelle Stand von Mapbender\,3 wird vorgestellt.
Zukünftige Entwicklungen werden diskutiert. Die Anwender können sich austauschen.
Alle Interessierten sind herzlich eingeladen.}
