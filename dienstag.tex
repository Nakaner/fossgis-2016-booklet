\renewcommand{\konferenztag}{\dienstag}
\newtimeslot{09:00}
\abstractGruen{Pirmin Kalberer}%
{GeoPackage, das Shapefile der Zukunft}%
{}%
{Im Februar 2014 hat das Open Geospatial Consortium (OGC) den GeoPackage Encoding Standard offiziell
freigegeben. Dieses noch junge Format hat sich bereits gut etabliert und wird in zahlreichen GIS-Produkten unterstützt.
In GeoPackage-Dateien können sowohl Vektor- als auch Rasterdaten mit zugehörigen Metainformation gespeichert
werden. Damit können Geodaten einfach ausgetauscht und auch auf mobilen Geräten effizient genutzt werden.}

\abstractGiStudio{Christopher Lorenz}%
{osm\_address\_db - Adressdaten in der OSM-Datenbank}%
{}%
{Das Projekt osm\_address\_db ermöglicht die Aufbereitung der in OSM vorhandenen Adressdaten. Dazu wurden Shell-
und SQL-Skripte entwickelt, die eine Tabellenstruktur zur Auswertung und Analyse aufbauen.
Im Vortrag wird die Datenstruktur, benutzten Tools und Technologien von osm\_address\_db als auch die während der
Entwicklung aufgetretenen Probleme eingegangen. Die OSM-Daten werden nach dem Karlsruher Schema analysiert,
dabei finden auch associatedStreet-Relationen Berücksichtigung.
Derzeit befindet sich eine Webanwendung zur Auswertung in Entwicklung.}

\newtimeslot{09:30}
\abstractGruen{Michael Paulmann}%
{Umstellung von Übungen auf QGIS}%
{}%
{Umstellung des Hochschulmoduls Geographische Informationssysteme von proprietärer Software auf die Open Source Software QGIS. Es gibt einen Einblick in die Probleme die bisher auftraten und den bisherigen Fortschritt. }

\abstractGiStudio{Manuel Roth}%
{Vector Tiles from OpenStreetMap}%
{}%
{Das Projekt OSM2VectorTiles bietet einerseits einen Workflow um selbst Vektor Tiles aus OpenStreetMap zu erstellen und bietet diese andererseits gratis zum Download an. Dies ermöglicht jedem eine selbst gestaltete Karte der gesamten Welt zu erstellen und anzubieten, ohne Kenntnisse von PostGIS und Mapnik zu haben.
Die Zuhörer wissen nach unserer Präsentation wie sie eine selbst gestaltete Karte erstellen und veröffentlichen können. Zudem können sie die vom OSM2VectorTiles Projekt zur Verfügung gestellten Daten verwenden, um ihre Karte auch Offline anbieten zu können.}

\newtimeslot{10:00}

\abstractGruen{Manfred Egger}%
{Automatische Erkennung der Projektion von Geodaten}%
{}%
{Das GIS-Programm SHAPEFILE PROJECTIONFINDER wird als  Lösungsvorschlag in Zusammenhang mit Geodaten und
unbekannter Projektion präsentiert. Zielgruppe dieses Programms sind vor allem Anwender, die mit geringen
Fachkenntnissen zu Koordinatensystemen ohne Aufwand gelieferte Daten in GIS-Projekte lagerichtig integrieren möchten.
Das Programm wertet ein beliebiges Shapefile mit unbekannter Projektion aus und schlägt auf Basis vorhandener offener
Onlinedienste mögliche zutreffende Projektionen vor, die der Anwender automatisch zuweisen kann. Der Aufwand bei unklarer
Projektion wird so reduziert.}

\abstractGiStudio{Dietmar Seifert}%
{Stand der Hausnummern in OSM und Hausnummerauswertung auf regio-osm.de}%
{}%
{}

\newtimeslot{11:00}

\abstractGruen{Andreas Schmid}%
{Betrieb von QGIS in einer heterogenen Client-Server-Umgebung}%
{}%
{Das Amt für Geoinformation des Kantons Solothurn (Schweiz) betreibt für die Benutzer der Kantonsverwaltung QGIS auf einem Linux-Applikationsserver. Für die Integration der Linux-Anwendung in den via Citrix bereitgestellten Windows-Desktop, der den Standard-Arbeitsplatz in der Kantonsverwaltung darstellt, kommt die Open-Source-Lösung X2Go zum Einsatz.
Der Vortrag präsentiert diese Lösung im Detail und geht auf deren Vor- und Nachteile ein. Zudem vergleicht er sie mit anderen Lösungen, die früher im Einsatz waren und nun abgelöst wurden.}

\abstractGiStudio{Lars Schimmer}%
{Mapillary - Alltag}%
{}%
{Ein Userbericht aus 18 Monaten Nutzung des Mapillary Ökosystems }

\newtimeslot{11:30}

\abstractGruen{Sara Biesel}%
{QGIS meets MapProxy}%
{}%
{}

\abstractGiStudio{Serhan Şen}%
{Leitstellensimulator goes OpenStreetMap}%
{}%
{Das nichtkommerzielle Browserspiel LstSim, ein Rettungsleitstellensimulator, setzte jahrelang auf die Dienste
der Google Maps API. Leider gab es mit der API über einen längeren Zeitraum anhaltende Schwierigkeiten. Es wurde
deshalb schon früh der Vorschlag gemacht, man solle doch "einfach auf OpenStreetMap wechseln". Doch was bedeutet
das überhaupt und wie sieht so ein Wechsel aus? Dieser Vortrag fasst die technischen und organisatorischen
Herausforderungen – und Chancen – seit Beginn der Umstellungen Ende 2015 zusammen.}

\newtimeslot{12:00}
\abstractGruen{Jörg Habenicht}%
{Die Zugriffszeit auf den QGis-Mapserver um Faktor 100 beschleunigen.}%
{}%
{}

\abstractGiStudio{Numa Gremling}%
{Turf.js – Geoverarbeitung im Browser}%
{}%
{Turf.js ist eine Open Source JavaScript-Bibliothek, die geographische Analysen und Abfragen ermöglicht.
Im Gegensatz zu Web Processing Services (WPS), die eine komplexe serverseitige Infrastruktur erfordern,
arbeitet Turf.js clientseitig mit Daten im GeoJSON-Format. Somit ist Turf.js eine schnell und leicht
umzusetzende Alternative zu komplexen Web-GIS-Lösungen.
Im Vortrag werden klassische Werkzeuge zur Analyse von Geodaten, wie Puffer und Punkt-in-Polygon, aber auch
die Durchführung komplexer Analysen vorgestellt.}

\newtimeslot{13:30}
\abstractGruen{Jonas Eberle}%
{Web-basierte Geoprozessierung mit Python und PyWPS}%
{}%
{Web-basierte Prozessierungsdienste können für eine Vielzahl an Aufgaben verwendet werden und bieten enorme Möglichkeiten,
Nutzer- und Entwicklerfreundliche Geodatendienste aufzubauen. Dies wird an einigen einfachen und komplexen Beispielen gezeigt,
die mit der Software PyWPS im Rahmen der Programmiersprache Python online veröffentlicht worden sind.
Dieser Vortrag liefert anfangs eine Einführung in den Bereich der „Web Processing Services“ und zeigt
zudem das Potential web-basierter Prozessierung mittels WPS im Geodatenbereich auf. }

\abstractGiStudio{Frederik Ramm}%
{OpenStreet mal ohne Map}%
{}%
{Dieser Vortrag erzählt von verschiedenen - realistischen wie auch unrealistischen - Anforderungen,
die potentielle Nutzer abseits von der Kartenerstellung an OpenStreetMap stellen, und skizziert Lösungen und Probleme.}

\newtimeslot{14:00}
\abstractGruen{Arne Schubert}%
{Hybride mobile App-Entwicklung mit Angular}%
{}%
{}

\abstractGiStudio{Joachim Kast}%
{Bezahlte und organisierte Edits - Vorteile und Gefahren für OSM}%
{}%
{}

\newtimeslot{14:30}
\abstractGruen{Florian Ledermann}%
{mapmap.js - Ein kartographisches API für interaktive thematische Karten}%
{}%
{Das an der TU Wien entwickelte API mapmap.js versucht den kartographischen Visualisierungsprozess in seiner Gesamtheit in einem high-level JavaScript API abzubilden, wobei jeder Teilaspekt zunächst „einfach funktioniert“, aber bei Bedarf im Detail angepasst werden kann. Die kartographische Visualisierungspipeline – vom Laden der Daten, über das Zuweisen graphischer Repräsentationen bis zur User-Interaktion – erlaubt die Erstellung von einfachen interaktiven Karten in wenigen Zeilen, aber auch die Anpassung im Detail und somit die Entwicklung völlig neuer Visualisierungstechniken.}

\abstractGiStudio{Sven Geggus}%
{Ansätze zur Lokalisierung einer Openstreetmap basierten Weltkarte}%
{}%
{Der Vortrag stellt die Implementierung von Lokalisierungsfunktionen als sogenannte "`stored procedures"' in PostrgreSQL vor,
die zur Latinisierung des deutschen Kartenstils eingesetzt wird.

Als Datenquelle dient die OSM-Datenbank selbst, die bereits viele lokalisierte Namen enthält. Fehlen diese erfolgt alternativ
die automatisierte Umschrift (Transliteration).

Die Transliteration birgt jedoch eine Vielzahl von Problemen, die schrift- oder sprachabhängige Lösungen benötigen. Der Vortrag
stellt sowohl die Implementierung solcher Lösungen als auch Probleme, die nur schwer lösbar erscheinen, vor.}

\newtimeslot{15:30}
\abstractGruen{Marc Jansen}%
{OpenLayers 3: Stand, Neues und Ausblick}%
{}%
{}

\abstractGiStudio{TBA}%
{OSM Lightning Talks}%
{}%
{}

\newtimeslot{16:00}
\abstractGruen{Tobias Sauerwein}%
{Faster, smaller, better: Compiling your application together with OpenLayers 3}%
{}%
{}

\abstractGiStudio{Lars Roskoden}%
{Das ist ja wohl die Höhe!}%
{}%
{Darstellung einfacher Möglichkeiten für die Höhenmessung von OSM-Objekten. Vergleich der Vor- und Nachteile.}

\newtimeslot{16:30}
\abstractGruen{Marc Jansen}%
{GeoExt3}%
{}%
{}

\abstractGiStudio{Peter Karich}%
{Flexible Routenplanung mit GraphHopper}%
{}%
{GraphHopper ist ein schneller und flexibler Open Source Routenplaner basierend auf OpenStreetMap Daten,
der sowohl offline auf dem Gerät als auch auf dem Server läuft und schon bei vielen bekannten Organisationen
und Firmen produktiv eingesetzt wird. GraphHopper ermöglicht nicht nur Routing von A nach B, sondern auch
Reichweitenanalyse, Map Matching und vieles mehr.}

\newtimeslot{17:00}
\abstractGruen{Astrid Emde}%
{Datenerfassung und Suchen mit Mapbender3}%
{}%
{Der Vortrag zu Mapbender3 stellt fortgeschrittene Elemente mit erweiterter Konfiguration vor. 
\begin{itemize}
 \item Möglichkeiten der Datenerfassung mit dem Mapbender3 Digitizer
 \item Aufbau von Suchen mit dem SearchRouter (SQL basierte Suchen)
 \item Aufbau von Suchen mit SimpleSearch (Solr basierte Suchen)
\end{itemize}%
}

\abstractGiStudio{Bernhard Ströbl}%
{XPlanung für einen Flächennutzungsplan mit PostGIS und QGIS}%
{}%
{Der Vortrag zeigt die erfolgte Umsetzung des Standards XPlanung für PostGIS und den Zugriff darauf aus
QGIS heraus am Beispiel eines in der Aufstellung befindlichen Flächennutzungsplans. }


