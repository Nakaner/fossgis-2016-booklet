\renewcommand{\konferenztag}{\dienstag}
\newtimeslot{09:00}
\abstractGruen{Pirmin Kalberer}%
{GeoPackage, das Shapefile der Zukunft}%
{}%
{Im Februar 2014 hat das Open Geospatial Consortium den GeoPackage Encoding Standard offiziell
freigegeben. Dieses noch junge Format hat sich bereits gut etabliert und wird in zahlreichen GIS-Produkten unterstützt.
In Geo\-Package-Dateien können sowohl Vektor- als auch Rasterdaten mit zugehörigen Metainformation gespei\-chert
werden. Damit können Geodaten einfach ausgetauscht und auch auf mobilen Geräten effizient genutzt werden.}

\abstractGiStudio{Christopher Lorenz}%
{osm\_address\_db -- Adressdaten in der OSM-Datenbank}%
{}%
{Das Projekt osm\_address\_db ermöglicht die Aufbereitung der in OSM vorhandenen Adressdaten. Dazu wurden Shell-
und SQL-Skripte entwickelt, die eine Tabellenstruktur zur Auswertung und Analyse aufbauen.
Im Vortrag wird die Datenstruktur, die benutzten Tools und Technologien als auch die während der
Entwicklung aufgetretenen Probleme eingegangen. Die OSM-Daten werden nach dem Karlsruher Schema analysiert,
dabei finden auch associated\-Street-Relationen Berücksichtigung.
Derzeit befindet sich eine Webanwendung zur Auswertung in Entwicklung.}

\newtimeslot{09:30}
\abstractGruen{Michael Paulmann}%
{Umstellung von Übungen auf QGIS}%
{}%
{Im Modul Grundlagen Geographischer Informationssysteme an der Hochschule für Technik in Stuttgart
sollen die Übungen und die Vorlesung nicht mehr in einem proprietären Geoinformationssystem
stattfinden, sondern in die Open-Source-Software QGIS. Es gibt einen Einblick in die Probleme
die bisher auftraten und den bisherigen Fortschritt, der Prozess ist noch nicht abgeschlossen.
Bis Oktober 2016 soll alles auf QGIS geändert worden sein.}

\abstractGiStudio{Manuel Roth}%
{Vector Tiles from OpenStreetMap}%
{}%
{Das Projekt OSM2VectorTiles bietet einerseits einen Workflow, um selbst Vektor Tiles aus OSM zu erstellen,
und bietet diese andererseits gratis zum Download an. Dies ermöglicht jedem, eine selbst gestaltete Karte der
gesamten Welt zu erstellen und anzubieten, ohne Kenntnisse von PostGIS und Mapnik zu haben.
Die Zuhörer wissen nach unserer Präsentation, wie sie eine selbst gestaltete Karte erstellen
und veröffentlichen können. Zudem können sie die vom OSM2VectorTiles-Projekt zur Verfügung gestellten
Daten verwenden, um ihre Karte auch offline anbieten zu können.}

\newtimeslot{10:00}

\abstractGruen{Manfred Egger}%
{Automatische Erkennung der Projektion von Geodaten}%
{}%
{Das GIS-Tool SHAPEFILE PROJECTIONFINDER wird als Lösungsvorschlag
in Zusammenhang mit Geodaten und
unbekannter Projektion präsentiert. Zielgruppe dieses Programms sind vor
allem Anwender, die mit geringen
Fachkenntnissen zu Koordinatensystemen ohne Aufwand gelieferte Daten in
GIS-Projekte lagerichtig integrieren möchten.}

\abstractGiStudio{Dietmar Seifert}%
{Stand der Hausnummern in OSM und Hausnummerauswertung auf regio-osm.de}%
{}%
{}

\newtimeslot{11:00}

\abstractGruen{Andreas Schmid}%
{Betrieb von QGIS in einer heterogenen Client-Server-Umgebung}%
{}%
{Das Amt für Geoinformation des Kantons Solothurn (Schweiz) betreibt für die Benutzer der Kantonsverwaltung QGIS auf einem Linux-Applikationsserver. Für die Integration der Linux-Anwendung in den via Citrix bereitgestellten Windows-Desktop, der den Standard-Arbeitsplatz in der Kantonsverwaltung darstellt, kommt die Open-Source-Lösung X2Go zum Einsatz.
Der Vortrag präsentiert diese Lösung im Detail und geht auf deren Vor- und Nachteile ein. Zudem vergleicht er sie mit anderen Lösungen, die früher im Einsatz waren und nun abgelöst wurden.}

\abstractGiStudio{Lars Schimmer}%
{Mapillary - Alltag}%
{}%
{Das Ökosystem Mapillary wurde auf der Fossgis 2014 in einem Lightning
Talk schon vorgestellt.
Ich werde aus Usersicht meine Erfahrungen aus 18 Monaten Aktivität mit
Mapillary berichten, dieses umfasst über 160\,000 eingereichte Fotos,
diverse genutzte Mobilgeräte, viele besuchte Orte und mehrere Methoden
der Datenbearbeitung. Natürlich werden auch die Pitfalls und Probleme angesprochen, über die
man zwangsweise stolpert}

\newtimeslot{11:30}

\abstractGruen{Sara Biesel}%
{QGIS meets MapProxy}%
{}%
{Das BKG entwickelt ein System, um Kartenausschnitte zu speichern und
die Karten Offline im Geoinformatiomssystem QGIS zu laden. Dabei wird ein
Proxyserver "`MapProxy"' verwendet, der bei bestehender Internetverbindung
die Karten als Kacheln lokal ablegt. Das ganze System liegt auf einer
portablen, externen Festplatte.}

\abstractGiStudio{Serhan Şen}%
{Leitstellensimulator goes OpenStreetMap}%
{}%
{Das nichtkommerzielle Browserspiel LstSim, ein Rettungsleitstellensimulator, setzte jahrelang auf die Dienste
der Google Maps API. Leider gab es mit der API über einen längeren Zeitraum anhaltende Schwierigkeiten. Es wurde
deshalb schon früh der Vorschlag gemacht, man solle doch "einfach auf OpenStreetMap wechseln". Doch was bedeutet
das überhaupt und wie sieht so ein Wechsel aus? Dieser Vortrag fasst die technischen und organisatorischen
Herausforderungen – und Chancen – seit Beginn der Umstellungen Ende 2015 zusammen.}

\newtimeslot{12:00}
\abstractGruen{Jörg Habenicht}%
{Die Zugriffszeit auf den QGis-Mapserver um Faktor 100 beschleunigen.}%
{}%
{In unserem Setup wird der QGis-Mapserver verwendet, um die Sessions von QGis
über ein Webinterface anzuzeigen. Die QGis Konfigurationsdateien haben eine
Speichergröße von bis zu 19MB, was in der QGis-Version 2.10 den Zugriff
auf das Webinterface auf ca.
30 Sekunden verlängert. 30 Sekunden ist auch der Timeout des Webbrowsers,
so dass es in unregelmäßigen Abständen Sessionabbrüche gibt.


Wir haben ein Tool programmiert, mit dem die Sessionabbrüche verhindert und
die Zugriffszeiten des qgis-servers Version 2.10 um den Faktor 100
beschleunigt werden. Das Programm nutzt intensiv die Caches des
Qgis-Servers aus, indem die Zugriffe für eine Karte nur einem bestimmten
Prozess zur Bearbeitung zugeteilt werden.
Das Programm wird mit dem Vortrag vorgestellt, zusammen mit Erklärungen
für den Aufbau des Programms, der Funktionsweise, der Setupkonfiguration
für einen Server und ein paar Messdaten.

Der Quellcode des Tools steht auf GitHub unter
https://github.com/geocalc/qgis-scheduler zur Verfügung.}

\abstractGiStudio{Numa Gremling}%
{Turf.js – Geoverarbeitung im Browser}%
{}%
{Turf.js ist eine Open Source JavaScript-Bibliothek, die geographische Analysen und Abfragen ermöglicht.
Im Gegensatz zu Web Processing Services (WPS), die eine komplexe serverseitige Infrastruktur erfordern,
arbeitet Turf.js clientseitig mit Daten im GeoJSON-Format. Somit ist Turf.js eine schnell und leicht
umzusetzende Alternative zu komplexen Web-GIS-Lösungen.
Im Vortrag werden klassische Werkzeuge zur Analyse von Geodaten, wie Puffer und Punkt-in-Polygon, aber auch
die Durchführung komplexer Analysen vorgestellt.}

\newtimeslot{13:30}
\abstractGruen{Jonas Eberle}%
{Web-basierte Geoprozessierung mit Python und PyWPS}%
{}%
{Web-basierte Prozessierungsdienste können für eine Vielzahl an Aufgaben verwendet werden und bieten enorme Möglichkeiten,
Nutzer- und Entwicklerfreundliche Geodatendienste aufzubauen. Dies wird an einigen einfachen und komplexen Beispielen gezeigt,
die mit der Software PyWPS im Rahmen der Programmiersprache Python online veröffentlicht worden sind.
Dieser Vortrag liefert anfangs eine Einführung in den Bereich der „Web Processing Services“ und zeigt
zudem das Potential web-basierter Prozessierung mittels WPS im Geodatenbereich auf. }

\abstractGiStudio{Frederik Ramm}%
{OpenStreet mal ohne Map}%
{}%
{Dieser Vortrag erzählt von verschiedenen - realistischen wie auch unrealistischen - Anforderungen,
die potentielle Nutzer abseits von der Kartenerstellung an OpenStreetMap stellen, und skizziert Lösungen und Probleme.}

\newtimeslot{14:00}
\abstractGruen{Arne Schubert}%
{Hybride mobile App-Entwicklung mit Angular}%
{}%
{}

\abstractGiStudio{Joachim Kast}%
{Bezahlte und organisierte Edits - Vorteile und Gefahren für OSM}%
{}%
{}

\newtimeslot{14:30}
\abstractGruen{Florian Ledermann}%
{mapmap.js - Ein kartographisches API für interaktive thematische Karten}%
{}%
{Das an der TU Wien entwickelte API mapmap.js versucht den kartographischen
Visualisierungsprozess in seiner Gesamtheit in einem high-level JavaScript API
abzubilden, wobei jeder Teilaspekt zunächst „einfach funktioniert“, aber bei
Bedarf im Detail an die Notwendigkeiten der jeweiligen Anwendung angepasst werden
kann und somit auch die Entwicklung neuer Visualisierungstechniken erlaubt.}

\abstractGiStudio{Sven Geggus}%
{Ansätze zur Lokalisierung einer Openstreetmap basierten Weltkarte}%
{}%
{Der Vortrag stellt eine vom Renderer unabhängige Methode zur Latinisierung
von OSM-basierten Karten vor. Als Datenquelle dient, wenn möglich, OSM
selbst. Alternativ wird Transkription verwendet, die jedoch viele
Schriftsystem-abhängige Probleme birgt, für die teilweise Lösungen
aufgezeigt werden. Ferner wird auf politische Probleme bei der Lokalisierung
von Karten eingegangen.}

\newtimeslot{15:30}
\abstractGruen{Marc Jansen}%
{OpenLayers 3: Stand, Neues und Ausblick}%
{}%
{}

\abstractGiStudio{TBA}%
{OSM Lightning Talks}%
{}%
{}

\newtimeslot{16:00}
\abstractGruen{Tobias Sauerwein}%
{Faster, smaller, better: Compiling your application together with OpenLayers 3}%
{}%
{}

\abstractGiStudio{Lars Roskoden}%
{Das ist ja wohl die Höhe!}%
{}%
{Darstellung einfacher Möglichkeiten für die Höhenmessung von OSM-Objekten. Vergleich der Vor- und Nachteile.}

\newtimeslot{16:30}
\abstractGruen{Marc Jansen}%
{GeoExt3}%
{}%
{}

\abstractGiStudio{Peter Karich}%
{Flexible Routenplanung mit GraphHopper}%
{}%
{GraphHopper ist ein schneller und flexibler Open Source Routenplaner basierend auf OpenStreetMap Daten,
der sowohl offline auf dem Gerät als auch auf dem Server läuft und schon bei vielen bekannten Organisationen
und Firmen produktiv eingesetzt wird. GraphHopper ermöglicht nicht nur Routing von A nach B, sondern auch
Reichweitenanalyse, Map Matching und vieles mehr.}

\newtimeslot{17:00}
\abstractGruen{Astrid Emde}%
{Datenerfassung und Suchen mit Mapbender3}%
{}%
{Der Vortrag zu Mapbender3 stellt fortgeschrittene Elemente mit erweiterter Konfiguration vor. 
\begin{itemize}
 \item Möglichkeiten der Datenerfassung mit dem Mapbender3 Digitizer
 \item Aufbau von Suchen mit dem SearchRouter (SQL basierte Suchen)
 \item Aufbau von Suchen mit SimpleSearch (Solr basierte Suchen)
\end{itemize}%
}

\abstractGiStudio{Bernhard Ströbl}%
{XPlanung für einen Flächennutzungsplan mit PostGIS und QGIS}%
{}%
{Der Vortrag zeigt die erfolgte Umsetzung des Standards XPlanung für PostGIS und den Zugriff darauf aus
QGIS heraus am Beispiel eines in der Aufstellung befindlichen Flächennutzungsplans. }


