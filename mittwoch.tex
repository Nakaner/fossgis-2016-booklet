\renewcommand{\konferenztag}{\mittwoch}
\newtimeslot{09:00}
\abstractGruen{Nikolaus Dürk}%
{Zusammenspiel von GIS und CMS verdeutlicht die möglichen Folgen einer 2\textdegree-Klimaerwärmung}%
{}%
{Die Linzer Unternehmen X-Net Services GmbH und blp GeoServices GmbH haben die technische Entwicklung
des öffentlich zugänglichen IMPACT2C-Web-Atlas im Auftrag des Climate Service Center Germany realisiert.
Darin werden die möglichen Auswirkungen einer 2\textdegree C"=Klimaerwärmung visualisiert und in einem Dual-Screen
(Map und Pages) für die Allgemeinheit verständlich erklärt. Durch die Darstellung in mehreren Layern
kann die Situation vor und nach einer 2\textdegree C-Klimaerwärmung verglichen werden.}

\enlargethispage{4ex}
\abstractGiStudio{Peter Barth}%
{Summer of Code}%
{}%
{Bereits seit 2008 nimmt das OSM-Projekt als Organisation am Google Summer of Code teil und
hat in diesem Rahmen viele interessante und nützliche Projekte auf den Weg gebracht, aber auch dazu
beigetragen die OSM"=Community zu vergrößern. Der Vortrag gibt einen Überblick über die
vergangenen Jahre, die geleistete Arbeit, die positiven wie negativen Aspekte der Teilnahme, zeigt
aber vor allem auch auf, warum sowohl Studenten als auch OSM von Google Summer of Code profitieren.}

\newtimeslot{09:30}
\abstractGruen{Daniel Koch}%
{Neuerungen im GeoServer}%
{}%
{\dots\ Die sehr aktive GeoServer Community arbeitet laufend an Erweiterungen und Verbesserungen
der Kernsoftware. Dieser Vortrag wird sich einigen (Neu"=)Entwicklungen der jüngeren
Vergangenheit widmen und an praktischen Beispielen den Nutzen vorstellen.

Hierunter fallen u.a. die Importer-Extension zum Hinzufügen von Geodaten in den GeoServer über das Webinterface,
die CSS-Extension zum Stylen von Layern über Cascading Style Sheets,
der WFS-Datenspeicher zur Kaskadierung entfernter WFS-Server,
die Darstellung von Curved Geometries und
die GeoFence-Integration.

Der Vortag wird mit einem Ausblick auf geplante und zukünftige Entwicklungen abschließen.}

\abstractGiStudio{Mark Padgham}%
{OSM schön gemacht}%
{}%
{Das R package, \emph{osmplotr} ermöglicht  die grafisch beliebige Darstellung ausgwählter OSM-Daten mit dem
wesentlichen Vorteil, dass ausgewählte Regionen mit Hilfe eines kontrastierenden Hintergrundes vergleichend
dargestellt werden können. Fokalbereiche können durch unterschiedliche osmoplotr-Funktionen bestimmt werden.}

\newpage
\vspace*{2\baselineskip}
% Text identisch mit 2015.
\sponsorenbox{103-norbit}{0.4\textwidth}{8}{%
	\textbf{Silbersponsor}\\
Als Softwarehaus für GIS-Kom\-plett\-lösungen entwickelt die norBIT GmbH seit über 25 Jahren mit norGIS 
modular aufgebaute Fachschalen für Anwendungen im Bereich von Tiefbau, Ver- und Entsorgung, die mit 
verschiedenen Grafikplattformen, wie QGIS, AutoCAD und BricsCAD betrieben werden (ggf. auch gemischt).
Seit über 8 Jahren beteiligt sich die norBIT GmbH an der Weiterentwicklung des QGIS und nutzt die 
Software zur Erfüllung verschiedenster Kundenanforderungen von Auskunftslösungen bis zu GIS-Arbeitsplätze.
Mit dem norGIS-ALKIS-Import und dem ALKIS-Plugin für QGIS hat die norBIT GmbH eine Open-Source-Lösung 
für die Verarbeitung und Visualisierung von ALKIS-Daten in QGIS und UMN Mapserver geschaffen.
}
\ifthispageodd{\ThisCenterWallPaper{1.0}{mittwoch-ungerade}}{\ThisCenterWallPaper{1.0}{mittwoch-gerade}}%

\newtimeslot{10:30}
\abstractGruen{Jürgen Weichand}%
{Neue Werkzeuge für INSPIRE}%
{}%
{Der INSPIRE-Zeitplan sieht die initiale Bereitstellung von INSPIRE-konformen
Daten -- für Themen aus Anhang I der Richtlinie -- bis spätestens November 2017 vor.
Für die Realisierung der benötigten Darstellungs- und Download\-dienste und die Durchführung
der benötigen Datenmodelltransformationen stehen immer geeignetere Verfahren und Werkzeuge
im FOSS-Bereich zur Verfügung.

So unterstützt beispielsweise HALE (The HUMBOLDT Alignment Editor) neuerdings die GeoServer-Erweiterung
für Applikationsschemata und ermöglicht hierdurch ein bequemes Mapping zwischen
bestehender Datenquelle und GML-Schema~\dots}

\enlargethispage{2\baselineskip}
\abstractGiStudio{Tobias Knerr}%
{OSM2World hinter den Kulissen}%
{}%
{Dieser Vortrag widmet sich den Herausforderungen, welche für die immer anspruchsvolleren 3D-Darstellungen in
OSM2World zu bewältigen waren. Der freie 3D-Renderer geht dabei oft einen anderen Weg als die Welt der 2D-Karten.

Zu den Themen zählen das Zeichnen von Küstenlinien, Straßenflächen und detaillierten Gebäuden ebenso wie neue
OpenGL-Funktionalitäten und Tagging-Schemata.}
%
\newtimeslot{11:00}
\abstractGruen{Armin Retterath}%
{INSPIRE "`instant"'}%
{}%
{Im Rahmen einer Live Präsentation wird gezeigt, wie sich frei verfügbare Geodaten auf einfache Weise und
in kürzester Zeit mittels QGIS Cloud und dem GeoPortal.rlp für INSPIRE bereitstellen lassen.
Das Verfahren kommt ohne den Betrieb eigener Webserver aus und eignet sich daher insbesondere
für kleinere Institutionen und Einrichtungen, die nur Daten von geringem Umfang verwalten.}
%
\abstractGiStudio{Arndt Brenschede}%
{Transit-Routing und OSM}%
{}%
{Transit-Routing ist in Bewegung und OSM spielt eine zunehmende Rolle dabei.
Was kann OSM ausser dem Wegenetz dazu beitragen, was ist der Unique-Selling Point
und warum ist das Thema mit Öffie, Google-Transit und der App vom Verkehrsverbund XY doch noch nicht entgültig besetzt?}

%
\newtimeslot{11:30}
\abstractGruen{Astrid Emde}%
{PostNAS-Suite}%
{}%
{Die PostNAS Suite bietet Lösungen zum Import von NAS Dateien und zur Weiterverarbeitung
sowie Inwertsetzung der Informationen. ALKIS, ATKIS, ABK werden im NAS Austauschformat
ausgegeben und können via ogr2ogr (GDAL-Bibliothek) in unterschiedliche Systeme
(PosgreSQL, Shape, Oracle u.\,a.) übertragen werden.
}

\abstractGiStudio{Philip Beelmann}%
{morituri}%
{}%
{Es gibt bereits viel freie Software, die mit Geodaten im OSM-Format umgehen kann.
Kommerzielle Geodaten hingegen können oft nur mit proprietärer Software genutzt werden.
Besitzer kommerzieller Geodaten hatten daher bisher keine Wahl.
In diesem Vortrag wird die freie Software "`morituri"' vorgestellt, die derzeit kommerzielle Geodaten von HERE in das OSM-Format konvertieren kann.
So kann man freie OSM-Software - wie beispielsweise die Routing-Engines Graphhopper oder OSRM, aber auch Geocoder wie Nominatim und Rendering-Software - zusammen mit den konvertierten kommerziellen Geodaten nutzen.}

\newpage
\sponsorenbox{210-beMasterGIS_final}{0.53\textwidth}{6}{%
\textbf{Bronzesponsor}\\
Online"=Masterstudiengang be\-Mas\-ter\-GIS\\
Fachanwender besit\-zen oft eine nur geringe GIS"=Ausbildung.
Seit 2010 hat sich der Online-Masterstudiengang GIS an der HS Anhalt etabliert.
GIS-Anwender aus Verwaltungs-, Planungs-, Umwelt- oder Marketingumfeld studieren
berufsbegleitend in fünf Semestern überwiegend online mit wenigen Präsenzphasen.}

\vfill
\ifthispageodd{\ThisCenterWallPaper{1.0}{mittwoch-ungerade}}{\ThisCenterWallPaper{1.0}{mittwoch-gerade}}
\sponsorenbox{201_Wheregroup}{0.48\textwidth}{3}{%
\RaggedRight\textbf{Bronzesponsor\\Stand im EXPO-Forum}\\
\justifying\noindent Die WhereGroup ist ein mittelständischer Lösungsanbieter im Bereich
webbasierter Geodateninfrastrukturen (GDI).
Zu unserem Portfolio gehören alle Dienstleistungen rund den Aufbau und Betrieb
dynamischer Kartenanwendungen mit Open-Source-Software sowie ein umfangreiches Schulungs- und Workshopprogramm.}

\vfill
