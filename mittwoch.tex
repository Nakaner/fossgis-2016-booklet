\renewcommand{\konferenztag}{\mittwoch}
\newtimeslot{09:00}
\abstractGruen{Nikolaus Dürk}%
{Zusammenspiel von GIS und CMS verdeutlicht die möglichen Folgen einer 2$^\circ$ Klimaerwärmung}%
{}%
{Die Linzer Unternehmen X-Net Services GmbH und blp GeoServices GmbH haben die technische Entwicklung
des öffentlich zugänglichen IMPACT2C Web-Atlas im Auftrag des Climate Service Center Germany realisiert.
Darin werden die möglichen Auswirkungen einer 2$^\circ$C Klimaerwärmung visualisiert und in einem Dual-Screen
(Map und Pages) für die Allgemeinheit verständlich erklärt. Durch die Darstellung in mehreren Layern
kann die Situation vor und nach einer 2$^\circ$C Klimaerwärmung verglichen werden.
Der IMPACT2C Web-Atlas wurde ausschließlich mit Open Source Komponenten realisiert.}

\abstractGiStudio{Peter Barth}%
{Summer of Code}%
{}%
{Bereits seit 2008 nimmt das OpenStreetMap-Projekt als Organisation am Google Summer of Code teil und
hat in diesem Rahmen viele interessante und nützliche Projekte auf den Weg gebracht, aber auch dazu
beigetragen die OpenStreetMap-Community zu vergrößern. Der Vortrag gibt einen Überblick über die
vergangenen Jahre, die geleistete Arbeit, die positiven wie negativen Aspekte der Teilnahme, zeigt
aber vor allem auch auf, warum sowohl Studenten als auch OpenStreetMap von Google Summer of Code profitieren.}

\newtimeslot{09:30}
\abstractGruen{Daniel Koch}%
{Neuerungen im GeoServer}%
{}%
{}

\abstractGiStudio{Mark Padgham}%
{OSM schön gemacht}%
{}%
{Das R package, "osmplotr" ermöglicht  die grafisch beliebige Darstellung ausgwählter OSM-Daten mit dem
wesentlichen Vorteil, dass ausgewählte Regionen mit Hilfe eines kontrastierenden Hintergrundes vergleichend
dargestellt werden können. Fokalbereiche können durch unterschiedliche osmoplotr-Funktionen bestimmt werden.}

\newtimeslot{10:30}
\abstractGruen{Jürgen Weichand}%
{Neue Werkzeuge für INSPIRE}%
{}%
{}

\abstractGiStudio{Tobias Knerr}%
{OSM2World hinter den Kulissen}%
{}%
{Dieser Vortrag widmet sich den Herausforderungen, die für die immer anspruchsvolleren 3D-Darstellungen in
OSM2World zu bewältigen waren. Der freie 3D-Renderer geht dabei oft einen anderen Weg als die Welt der 2D-Karten.

Zu den Themen zählen das Zeichnen von Küstenlinien, Straßenflächen und detaillierten Gebäuden ebenso wie neue
OpenGL-Funktionalitäten und Tagging-Schemata.}
%
\newtimeslot{11:00}
\abstractGruen{Armin Retterath}%
{INSPIRE "`instant"'}%
{}%
{Im Rahmen einer Live Präsentation wird gezeigt, wie sich frei verfügbare Geodaten auf einfache Weise und
in kürzester Zeit mittels QGIS Cloud und dem GeoPortal.rlp für INSPIRE bereitstellen lassen.
Das Verfahren kommt ohne den Betrieb eigener Webserver aus und eignet sich daher insbesondere
für kleinere Institutionen und Einrichtungen, die nur Daten von geringem Umfang verwalten.}
%
\abstractGiStudio{Arndt Brenschede}%
{Transit-Routing und OSM}%
{}%
{Transit-Routing ist in Bewegung und OSM spielt eine zunehmende Rolle dabei. Was kann OSM ausser dem Wegenetz dazu beitragen, was ist der Unique-Selling Point und warum ist das Thema mit Öffie, Google-Transit und der App vom Verkehrsverbund XY doch noch nicht entgültig besetzt?}
%
\newtimeslot{11:30}
\abstractGruen{Astrid Emde}%
{PostNAS-Suite}%
{}%
{}

\abstractGiStudio{Philip Beelmann}%
{morituri}%
{}%
{Es gibt bereits viel freie Software, die mit Geodaten im OSM-Format umgehen kann.
Kommerzielle Geodaten hingegen können oft nur mit proprietärer Software genutzt werden.
Besitzer kommerzieller Geodaten hatten daher bisher keine Wahl.
In diesem Vortrag wird die freie Software "morituri" vorgestellt, die derzeit kommerzielle Geodaten von HERE in das OSM-Format konvertieren kann.
So kann man freie OSM-Software - wie beispielsweise die Routing-Engines Graphhopper oder OSRM, aber auch Geocoder wie Nominatim und Rendering-Software - zusammen mit den konvertierten kommerziellen Geodaten nutzen.}


